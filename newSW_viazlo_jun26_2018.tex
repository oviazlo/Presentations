\documentclass[8pt]{beamer}

\newif\ifplacelogo % create a new conditional
\placelogotrue % set it to true

\usetheme{Warsaw}
\usecolortheme{rose}
\usepackage{multicol}
\usepackage{epstopdf}
\usepackage[italic]{hepnames}
\usepackage{tikz}
\usepackage{listings}
\usepackage{times}
\usepackage{amsmath}
\usepackage{verbatim}
\usepackage{hyperref}
\usepackage{bbding}
\usepackage{gensymb}
\usepackage{upgreek}
\lstset{breakatwhitespace,
language=C++,
columns=fullflexible,
keepspaces,
breaklines,
tabsize=3, 
showstringspaces=false,
extendedchars=true}

% TikZ includes!!!
\usepackage{tikz}
\usetikzlibrary{backgrounds}
\usetikzlibrary{calc}
\tikzstyle{every picture}+=[remember picture]
\input{/home/oviazlo/Desktop/beamerPresentations/myReports/latexHelpScripts/tikzGrid.tex}


\begin{document}

% custom colors
\definecolor{olive}{rgb}{0.3, 0.4, .1}
\definecolor{fore}{RGB}{249,242,215}
\definecolor{back}{RGB}{51,51,51}
\definecolor{title}{RGB}{255,0,90}
\definecolor{dgreen}{rgb}{0.,0.6,0.}
\definecolor{gold}{rgb}{1.,0.84,0.}
\definecolor{JungleGreen}{cmyk}{0.99,0,0.52,0}
\definecolor{BlueGreen}{cmyk}{0.85,0,0.33,0}
\definecolor{RawSienna}{cmyk}{0,0.72,1,0.45}
\definecolor{Magenta}{cmyk}{0,1,0,0}

\definecolor{PixelColor}{RGB}{207,232,139}
\definecolor{SCTColor}{RGB}{167,166,255}
\definecolor{TRTColor}{RGB}{250,224,140}
\definecolor{grayColor}{RGB}{153,153,153}

\newcommand{\yRefPosOne}{0.0}
\newcommand{\xRefPosOne}{0.0}
\newcommand{\yRefPosTwo}{0.0}
\newcommand{\xRefPosTwo}{0.0}
\newcommand{\yRefIncrementOne}{0.0}
\newcommand{\xRefIncrementOne}{0.0}
\newcommand{\yRefIncrementTwo}{0.0}
\newcommand{\xRefIncrementTwo}{0.0}

\graphicspath{ {/home/oviazlo/Desktop/beamerPresentations/FCCee/pictures/jun19_2018/} }


\DeclareGraphicsExtensions{.eps, .pdf, .png}

\newcommand{\myBox}[2][pink] {
    \noindent\colorbox{#1}{
	\textbf{#2}
    }\par
}

% For nice block (provided by Oleh)
\tikzstyle{myBox} = [draw=red, fill=blue!1, very thick,
    rectangle, rounded corners, inner sep=5pt, inner ysep=9pt]
    
\tikzstyle{PixelBox} = [draw=PixelColor, fill=blue!1, very thick,
    rectangle, rounded corners, inner sep=5pt, inner ysep=9pt]
\tikzstyle{SCTBox} = [draw=SCTColor, fill=blue!1, very thick,
    rectangle, rounded corners, inner sep=5pt, inner ysep=9pt]
\tikzstyle{TRTBox} = [draw=TRTColor, fill=blue!1, very thick,
    rectangle, rounded corners, inner sep=5pt, inner ysep=9pt]

% poster advertisement
\newcommand{\myCenterBox}[2][pink] {
   {\centering
    \noindent\colorbox{#1}{
	\textbf{#2}
    }\par
  }
}

\newcommand{\mySmallCenterBox}[2][pink] {
   {\centering
    \noindent\colorbox{#1}{
	\textbf{{\small #2}}
    }\par
  }
}

\newcommand{\myVerySmallCenterBox}[2][pink] {
   {\centering
    \noindent\colorbox{#1}{
	\textbf{{\scriptsize #2}}
    }\par
  }
}

\newcommand{\backupbegin}{
   \newcounter{finalframe}
   \setcounter{finalframe}{\value{framenumber}}
}
\newcommand{\backupend}{
   \setcounter{framenumber}{\value{finalframe}}
}

\newcommand{\myNode}{\tikz[baseline,inner sep=1pt] \node[anchor=base]}

\tikzstyle{fancytitle} =[fill=white!15, text=black]

\definecolor{light-gray}{gray}{0.95}
% poster advertisement


\title[Background overlay for CLD \hspace{14.0em}\insertframenumber/
\inserttotalframenumber]{ Background overlay for CLD}


	\author[Oleksandr Viazlo]{Oleksandr Viazlo\\ 
	{\small on behalf of the CLICdp and FCC-ee collaborations}
	}
	\institute{\small CERN\\} 
	
       
	\date{22 May 2018}

% 	\logo{ \ifplacelogo \includegraphics[height=1.8cm]{./ID_week2/lund_uni-logo_s.pdf} \hspace{0.4cm} \fi}

	
%    	\frame{\titlepage}

   	

\placelogofalse




%*****************************************************************************
\begin{frame}{\large \large Time window settings for background overlay processor for CLD detector}
\renewcommand{\yRefPosOne}{-0.9}
\renewcommand{\xRefPosOne}{4.2}
\renewcommand{\xRefIncrementOne}{7.5}
\begin{tikzpicture}[overlay]


 \node [PixelBox] at (\xRefPosOne+1,\yRefPosOne+3.1) (box){%
    \begin{minipage}{1.1\textwidth}
  \begin{itemize}
   \item VXD detector for CLD model is based on the ALICE ITS upgrade - 10$\mu$s readout window
    \item Tracking performance studies with background overlaid was done with settings:
   \begin{itemize}
    \item 400 ns (20 BX) for 91 GeV - limitation from cpu/memory usage
    \item 10$\mu$s (3BX) for 365 GeV \\[0.3cm]
   \end{itemize}
    \end{itemize}
    \end{minipage}
  };
\node[fancytitle, right=15pt] at (box.north west) {Tracking};

 \node [PixelBox] at (\xRefPosOne+1,\yRefPosOne+0.9) (box){%
    \begin{minipage}{1.1\textwidth}
  \begin{itemize}
   \item ECAL and HCAL detectors for CLD are identical to ones in CLIC - 10 ns integration window
   \item Allows to achieve a few ns precision for cluster timing
    \end{itemize}
    \end{minipage}
  };
\node[fancytitle, right=15pt] at (box.north west) {Calorimetery};
 
  \node [TRTBox] at (\xRefPosOne+1,\yRefPosOne-2.2) (box){%
    \begin{minipage}{1.1\textwidth}
  \begin{itemize}
   \item Pandora attempts to associate all tracks to calorimeter clusters
   \begin{itemize}
    \item tracks reconstructed in $\mu$s time window
    \item cluster reconstructed in ns time window
   \end{itemize}
   \item What is the best way to combine tracks and cluster in order to obtain the largest background rejection power?
   \item For jet performance studies one can increase integration time window for calorimeter to match the window used for tracking $\to$ decrease background rejection power (require recalibration of calorimeter)?
   

    \end{itemize}
    \end{minipage}
  };
\node[fancytitle, right=15pt] at (box.north west) {Jet performance with background overlay};
 
 \end{tikzpicture}
\end{frame}
%*****************************************************************************
% 
% \backupbegin
% %*****************************************************************************
% \begin{frame}
% \frametitle{BACKUP} 
%  
% \end{frame}
% %*****************************************************************************
% 
% \backupend
%********************************************************
\end{document}

