\documentclass[8pt]{beamer}

\newif\ifplacelogo % create a new conditional
\placelogotrue % set it to true

\usetheme{Warsaw}
\usecolortheme{rose}
\usepackage{multicol}
\usepackage{epstopdf}
\usepackage[italic]{hepnames}
\usepackage{tikz}
\usepackage{listings}
\usepackage{times}
\usepackage{amsmath}
\usepackage{verbatim}
\usepackage{hyperref}
\usepackage{bbding}
\usepackage{gensymb}
\usepackage{upgreek}
\lstset{breakatwhitespace,
language=C++,
columns=fullflexible,
keepspaces,
breaklines,
tabsize=3, 
showstringspaces=false,
extendedchars=true}

% TikZ includes!!!
\usepackage{tikz}
\usetikzlibrary{backgrounds}
\usetikzlibrary{calc}
\tikzstyle{every picture}+=[remember picture]
\input{/home/oviazlo/Desktop/beamerPresentations/myReports/latexHelpScripts/tikzGrid.tex}


\begin{document}

% custom colors
\definecolor{olive}{rgb}{0.3, 0.4, .1}
\definecolor{fore}{RGB}{249,242,215}
\definecolor{back}{RGB}{51,51,51}
\definecolor{title}{RGB}{255,0,90}
\definecolor{dgreen}{rgb}{0.,0.6,0.}
\definecolor{gold}{rgb}{1.,0.84,0.}
\definecolor{JungleGreen}{cmyk}{0.99,0,0.52,0}
\definecolor{BlueGreen}{cmyk}{0.85,0,0.33,0}
\definecolor{RawSienna}{cmyk}{0,0.72,1,0.45}
\definecolor{Magenta}{cmyk}{0,1,0,0}

\definecolor{PixelColor}{RGB}{207,232,139}
\definecolor{SCTColor}{RGB}{167,166,255}
\definecolor{TRTColor}{RGB}{250,224,140}
\definecolor{grayColor}{RGB}{153,153,153}

\newcommand{\yRefPosOne}{0.0}
\newcommand{\xRefPosOne}{0.0}
\newcommand{\yRefPosTwo}{0.0}
\newcommand{\xRefPosTwo}{0.0}
\newcommand{\yRefIncrementOne}{0.0}
\newcommand{\xRefIncrementOne}{0.0}
\newcommand{\yRefIncrementTwo}{0.0}
\newcommand{\xRefIncrementTwo}{0.0}

\graphicspath{ {/home/oviazlo/Desktop/beamerPresentations/FCCee/pictures/fromNote_Jul4/} }


\DeclareGraphicsExtensions{.eps, .pdf, .png}

\newcommand{\myBox}[2][pink] {
    \noindent\colorbox{#1}{
	\textbf{#2}
    }\par
}

% For nice block (provided by Oleh)
\tikzstyle{myBox} = [draw=red, fill=blue!1, very thick,
    rectangle, rounded corners, inner sep=5pt, inner ysep=9pt]
    
\tikzstyle{PixelBox} = [draw=PixelColor, fill=blue!1, very thick,
    rectangle, rounded corners, inner sep=5pt, inner ysep=9pt]
\tikzstyle{SCTBox} = [draw=SCTColor, fill=blue!1, very thick,
    rectangle, rounded corners, inner sep=5pt, inner ysep=9pt]
\tikzstyle{TRTBox} = [draw=TRTColor, fill=blue!1, very thick,
    rectangle, rounded corners, inner sep=5pt, inner ysep=9pt]

% poster advertisement
\newcommand{\myCenterBox}[2][pink] {
   {\centering
    \noindent\colorbox{#1}{
	\textbf{#2}
    }\par
  }
}

\newcommand{\mySmallCenterBox}[2][pink] {
   {\centering
    \noindent\colorbox{#1}{
	\textbf{{\small #2}}
    }\par
  }
}

\newcommand{\myVerySmallCenterBox}[2][pink] {
   {\centering
    \noindent\colorbox{#1}{
	\textbf{{\scriptsize #2}}
    }\par
  }
}

\newcommand{\backupbegin}{
   \newcounter{finalframe}
   \setcounter{finalframe}{\value{framenumber}}
}
\newcommand{\backupend}{
   \setcounter{framenumber}{\value{finalframe}}
}

\newcommand{\myNode}{\tikz[baseline,inner sep=1pt] \node[anchor=base]}

\tikzstyle{fancytitle} =[fill=white!15, text=black]

\definecolor{light-gray}{gray}{0.95}
% poster advertisement


\title[Calorimetry performance with CLD \hspace{14.0em}\insertframenumber/
\inserttotalframenumber]{ Calorimetry and PID performance with CLD }


	\author[Oleksandr Viazlo]{Oleksandr Viazlo\\ 
	{\small }
	}
	\institute{\small CERN\\} 
	
       
	\date{4 July 2018}

% 	\logo{ \ifplacelogo \includegraphics[height=1.8cm]{./ID_week2/lund_uni-logo_s.pdf} \hspace{0.4cm} \fi}

	
   	\frame{\titlepage}

   	

\placelogofalse

%*****************************************************************************
\begin{frame}{\large \large Photon energy resolution  and $K_L^0$ resolutions}

\renewcommand{\yRefPosOne}{-1.5}
\renewcommand{\xRefPosOne}{5.3}
\renewcommand{\xRefIncrementOne}{5.5}
\begin{tikzpicture}[overlay]

%   \node  at (\xRefPosOne+0.4,\yRefPosOne+4.5) (box){%
%     \begin{minipage}{1.1\textwidth}
%   \begin{itemize}
%   \item ???
%     \end{itemize}
%     \end{minipage}
%   };

 \node[inner sep=0pt] (tmp) at (\xRefPosOne-2.7,\yRefPosOne+2.0)
    {\includegraphics[width=6cm]{jun7_photonRes_barrel_vs_full_v2.pdf}};
    
 \node[inner sep=0pt] (tmp) at (\xRefPosOne+3.3,\yRefPosOne+2.0)
    {\includegraphics[width=6cm]{jun28_kaonRes_vs_energy.pdf}};
    
 
 \node  at (\xRefPosOne+0.1,\yRefPosOne-1.9) (box){%
    \begin{minipage}{1.1\textwidth}
  \begin{itemize}
  \item Photon energy resolution (left)  and neutral hadron resolutions of \PKzL's (right) as a function of energy in the barrel region, transition region and endcap.\\[+0.2cm]
  \item Energy response distributions are iteratively fitted with a Gaussian within a range $\pm3\sigma$\\[+0.2cm]
  
%   ECAL
%    1  p0           1.13097e-01   7.88838e-03   4.21011e-06   7.01453e-08
%    2  p1           1.51617e+01   4.74127e-02   4.91653e-05   1.14714e-08
%  HCAL barrel
%    1  p0           6.52800e-01   3.98079e-02   3.85090e-04  -2.95222e-09
%    2  p1           4.73880e+01   2.11919e-01   2.05004e-03  -5.54559e-10
% HCAL endcap
%    1  p0           7.82335e-01   7.56244e-02   3.29232e-04  -5.61127e-09
%    2  p1           4.52609e+01   4.06539e-01   1.76987e-03  -1.12410e-09



    \end{itemize}
    \end{minipage}
  };

% % HELPER draw advanced helping grid with axises:
% \draw(-0.5,-4) to[grid with coordinates] (11.5,4);
\end{tikzpicture}
 
\end{frame}
%*****************************************************************************

%*****************************************************************************
\begin{frame}{\large \large Muon identification efficiency}

\renewcommand{\yRefPosOne}{-1.5}
\renewcommand{\xRefPosOne}{5.3}
\renewcommand{\xRefIncrementOne}{5.5}
\begin{tikzpicture}[overlay]

%   \node  at (\xRefPosOne+0.4,\yRefPosOne+4.5) (box){%
%     \begin{minipage}{1.1\textwidth}
%   \begin{itemize}
%   \item ???
%     \end{itemize}
%     \end{minipage}
%   };

 \node[inner sep=0pt] (tmp) at (\xRefPosOne-2.7,\yRefPosOne+2.0)
    {\includegraphics[width=6cm]{single_particle_ID_efficiency/jun22_muons_eff_vs_energy.pdf}};
    
 \node[inner sep=0pt] (tmp) at (\xRefPosOne+3.3,\yRefPosOne+2.0)
    {\includegraphics[width=6cm]{single_particle_ID_efficiency/jun22_muons_eff_vs_cosTheta.pdf}};
    
 
 \node  at (\xRefPosOne+0.1,\yRefPosOne-1.9) (box){%
    \begin{minipage}{1.1\textwidth}
  \begin{itemize}
   \item Charged reconstructed particle required to satisfy following matching criteria  to ``true'' particle:
   \begin{itemize}
    \item the same PDG code\\[0.1cm]
    \item angular matching: $|\phi_{reco}-\phi_{true}|< 2$ mrad and $|\theta_{reco}-\theta_{     true}|< 1$ mrad\\[0.1cm]
    \item energy matching: $|p_T^{reco} - p_T^{true}| < 5\%$\\[0.2cm]
   \end{itemize}
  \item $>99\%$ muon efficiency for all energies
  
    \end{itemize}
    \end{minipage}
  };

% % HELPER draw advanced helping grid with axises:
% \draw(-0.5,-4) to[grid with coordinates] (11.5,4);
\end{tikzpicture}
 
\end{frame}
%*****************************************************************************

%*****************************************************************************
\begin{frame}{\large \large Pion identification efficiency}

\renewcommand{\yRefPosOne}{-1.5}
\renewcommand{\xRefPosOne}{5.3}
\renewcommand{\xRefIncrementOne}{5.5}
\begin{tikzpicture}[overlay]

%   \node  at (\xRefPosOne+0.4,\yRefPosOne+4.5) (box){%
%     \begin{minipage}{1.1\textwidth}
%   \begin{itemize}
%   \item ???
%     \end{itemize}
%     \end{minipage}
%   };

 \node[inner sep=0pt] (tmp) at (\xRefPosOne-2.7,\yRefPosOne+2.0)
    {\includegraphics[width=6cm]{single_particle_ID_efficiency/jun22_pions_eff_vs_energy.pdf}};
    
 \node[inner sep=0pt] (tmp) at (\xRefPosOne+3.3,\yRefPosOne+2.0)
    {\includegraphics[width=6cm]{single_particle_ID_efficiency/jun22_pions_eff_vs_cosTheta.pdf}};
    
 
 \node  at (\xRefPosOne+0.1,\yRefPosOne-1.3) (box){%
    \begin{minipage}{1.1\textwidth}
  \begin{itemize}
   \item $>90\%$ pion efficiency at low energies,  $>95\%$ starting from 20 GeV \\[0.2cm]
   \item more tight muon ID cuts are used to partially recover pion misidentification as muons
  
    \end{itemize}
    \end{minipage}
  };

% % HELPER draw advanced helping grid with axises:
% \draw(-0.5,-4) to[grid with coordinates] (11.5,4);
\end{tikzpicture}
 
\end{frame}
%*****************************************************************************

%*****************************************************************************
\begin{frame}{\large \large Electron identification efficiency}

\renewcommand{\yRefPosOne}{-1.5}
\renewcommand{\xRefPosOne}{5.3}
\renewcommand{\xRefIncrementOne}{5.5}
\begin{tikzpicture}[overlay]

%   \node  at (\xRefPosOne+0.4,\yRefPosOne+4.5) (box){%
%     \begin{minipage}{1.1\textwidth}
%   \begin{itemize}
%   \item ???
%     \end{itemize}
%     \end{minipage}
%   };

 \node[inner sep=0pt] (tmp) at (\xRefPosOne-2.7,\yRefPosOne+2.0)
    {\includegraphics[width=6cm]{single_particle_ID_efficiency/jun22_electrons_eff_vs_energy.pdf}};
    
 \node[inner sep=0pt] (tmp) at (\xRefPosOne+3.3,\yRefPosOne+2.0)
    {\includegraphics[width=6cm]{single_particle_ID_efficiency/jun22_electrons_eff_vs_cosTheta.pdf}};
    
 
 \node  at (\xRefPosOne+0.1,\yRefPosOne-1.9) (box){%
    \begin{minipage}{1.1\textwidth}
  \begin{itemize}
   \item Bremsstrahlung recovery algorithm:\\[0.1cm]
   \begin{itemize}
    \item preselect close by photons (within $|\phi_{reco}-\phi_{true}|< 20$ mrad and $|\theta_{reco}-\theta_{true}|< 1$ mrad)\\[0.1cm]
    \item dress the electron momentum by summing photon four momenta\\[0.1cm]
    \item impose looser energy matching requirement for dressed electrons \\(reconstructed energy within 5~$\sigma_{ECAL}$)\\[0.2cm]
   \end{itemize}
   \item $>95\%$ electron efficiency starting from 20 GeV \\[0.2cm]
  
    \end{itemize}
    \end{minipage}
  };

% % HELPER draw advanced helping grid with axises:
% \draw(-0.5,-4) to[grid with coordinates] (11.5,4);
\end{tikzpicture}
 
\end{frame}
%*****************************************************************************

%*****************************************************************************
\begin{frame}{\large \large Photon identification efficiency}

\renewcommand{\yRefPosOne}{-1.5}
\renewcommand{\xRefPosOne}{5.3}
\renewcommand{\xRefIncrementOne}{5.5}
\begin{tikzpicture}[overlay]

%   \node  at (\xRefPosOne+0.4,\yRefPosOne+4.5) (box){%
%     \begin{minipage}{1.1\textwidth}
%   \begin{itemize}
%   \item ???
%     \end{itemize}
%     \end{minipage}
%   };

 \node[inner sep=0pt] (tmp) at (\xRefPosOne-2.7,\yRefPosOne+2.0)
    {\includegraphics[width=6cm]{single_particle_ID_efficiency/jun22_photons_noConv_eff_vs_energy.pdf}};
    
 \node[inner sep=0pt] (tmp) at (\xRefPosOne+3.3,\yRefPosOne+2.0)
    {\includegraphics[width=6cm]{single_particle_ID_efficiency/jun22_photons_conv_eff_vs_energy.pdf}};
    
 
 \node  at (\xRefPosOne+0.1,\yRefPosOne-1.9) (box){%
    \begin{minipage}{1.1\textwidth}
  \begin{itemize}
    \item The same angular matching criteria
    \item Reconstructed energy has to be within 5~$\sigma_{ECAL}$ (5$\times0.15 \sqrt{E}$)
    \item $>99\%$ ID efficiency for unconverted photons (left plot)
    \item $>90\%$ ID efficiency starting from 50 GeV for converted photons with applying Bremsstrahlung recovery algorithm (right plot)
  
    \end{itemize}
    \end{minipage}
  };

% % HELPER draw advanced helping grid with axises:
% \draw(-0.5,-4) to[grid with coordinates] (11.5,4);
\end{tikzpicture}
 
\end{frame}
%*****************************************************************************

%*****************************************************************************
\begin{frame}{\large \large Jet Energy Resolution}

\renewcommand{\yRefPosOne}{-1.5}
\renewcommand{\xRefPosOne}{5.3}
\renewcommand{\xRefIncrementOne}{5.5}
\begin{tikzpicture}[overlay]

%   \node  at (\xRefPosOne+0.4,\yRefPosOne+4.5) (box){%
%     \begin{minipage}{1.1\textwidth}
%   \begin{itemize}
%   \item ???
%     \end{itemize}
%     \end{minipage}
%   };

 \node[inner sep=0pt] (tmp) at (\xRefPosOne-2.7,\yRefPosOne+2.0)
    {\includegraphics[width=6cm]{jun29_Zuds91_vs_Zuds365.pdf}};
    
 \node[inner sep=0pt] (tmp) at (\xRefPosOne+3.3,\yRefPosOne+2.0)
    {\includegraphics[width=6cm]{JER_SWC_noWC_vs_energy_res7.pdf}};
    
 
 \node  at (\xRefPosOne+0.1,\yRefPosOne-1.9) (box){%
    \begin{minipage}{1.1\textwidth}
  \begin{itemize}
   \item Jet energy resolution as function of $\cos \theta$ at 91 and 365 GeV energy\\[0.1cm]
   \item Resolution: 4.5-5$\%$ for 45.5 GeV jets, 3-4$\%$ for 182.5 GeV jets\\[0.1cm]
   \item Applying software compensation improves the energy resolution of jets  by 5-7$\%$\\[0.1cm]
  
    \end{itemize}
    \end{minipage}
  };

% % HELPER draw advanced helping grid with axises:
% \draw(-0.5,-4) to[grid with coordinates] (11.5,4);
\end{tikzpicture}
 
\end{frame}
%*****************************************************************************

%*****************************************************************************
\begin{frame}{\large \large Outlook}

\renewcommand{\yRefPosOne}{-1.5}
\renewcommand{\xRefPosOne}{5.3}
\renewcommand{\xRefIncrementOne}{5.5}
\begin{tikzpicture}[overlay]

%   \node  at (\xRefPosOne+0.4,\yRefPosOne+4.5) (box){%
%     \begin{minipage}{1.1\textwidth}
%   \begin{itemize}
%   \item ???
%     \end{itemize}
%     \end{minipage}
%   };
    
 \node[inner sep=0pt] (tmp) at (\xRefPosOne-2.6,\yRefPosOne+1.4)
    {\includegraphics[width=5cm]{../jun19_2018/muonEff_bbbar_FCCee/muonEff_bbar_vs_vertexR.pdf}};
    
%    \node  at (\xRefPosOne+3.5,\yRefPosOne+3.2) (box){%
%     \begin{minipage}{0.6\textwidth}
%   \begin{itemize}
%   \small
%    \item[*] Conformal tracking has difficulties to 
%    \item[*] 
%      \end{itemize}
%     \end{minipage}
%   };
 
 \node  at (\xRefPosOne+0.5,\yRefPosOne+4.2) (box){%
    \begin{minipage}{1.1\textwidth}
  \begin{itemize}
   \item Muon efficiency with $b\bar{b}$ sample at 365 GeV - seems like tracking-related issue \\[0.3cm]
     \end{itemize}
    \end{minipage}
  };

  \node  at (\xRefPosOne+0.5,\yRefPosOne-2.0) (box){%
    \begin{minipage}{1.1\textwidth}
  \begin{itemize}
   \item Lepton identification efficiency in $t\bar{t}$ events - work in progress
   \item Background contribution to calorimeter energy measurements as a function of time integration window
     \end{itemize}
    \end{minipage}
  };
  
% % HELPER draw advanced helping grid with axises:
% \draw(-0.5,-4) to[grid with coordinates] (11.5,4);
\end{tikzpicture}
 
\end{frame}
%*****************************************************************************

\backupbegin
%*****************************************************************************
\begin{frame}
\frametitle{BACKUP} 
 
\end{frame}
%*****************************************************************************

\backupend
%********************************************************
\end{document}

