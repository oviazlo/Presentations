\documentclass[8pt]{beamer}

\newif\ifplacelogo % create a new conditional
\placelogotrue % set it to true

\usetheme{Warsaw}
\usecolortheme{rose}
\usepackage{multicol}
\usepackage{epstopdf}
\usepackage[italic]{hepnames}
\usepackage{tikz}
\usepackage{listings}
\usepackage{times}
\usepackage{amsmath}
\usepackage{verbatim}
\usepackage{hyperref}
\usepackage{bbding}
\usepackage{gensymb}
\usepackage{upgreek}
\lstset{breakatwhitespace,
language=C++,
columns=fullflexible,
keepspaces,
breaklines,
tabsize=3, 
showstringspaces=false,
extendedchars=true}

% TikZ includes!!!
\usepackage{tikz}
\usetikzlibrary{backgrounds}
\usetikzlibrary{calc}
\tikzstyle{every picture}+=[remember picture]
\input{/home/oviazlo/Desktop/beamerPresentations/myReports/latexHelpScripts/tikzGrid.tex}


\begin{document}

% custom colors
\definecolor{olive}{rgb}{0.3, 0.4, .1}
\definecolor{fore}{RGB}{249,242,215}
\definecolor{back}{RGB}{51,51,51}
\definecolor{title}{RGB}{255,0,90}
\definecolor{dgreen}{rgb}{0.,0.6,0.}
\definecolor{gold}{rgb}{1.,0.84,0.}
\definecolor{JungleGreen}{cmyk}{0.99,0,0.52,0}
\definecolor{BlueGreen}{cmyk}{0.85,0,0.33,0}
\definecolor{RawSienna}{cmyk}{0,0.72,1,0.45}
\definecolor{Magenta}{cmyk}{0,1,0,0}

\definecolor{PixelColor}{RGB}{207,232,139}
\definecolor{SCTColor}{RGB}{167,166,255}
\definecolor{TRTColor}{RGB}{250,224,140}
\definecolor{grayColor}{RGB}{153,153,153}

\newcommand{\yRefPosOne}{0.0}
\newcommand{\xRefPosOne}{0.0}
\newcommand{\yRefPosTwo}{0.0}
\newcommand{\xRefPosTwo}{0.0}
\newcommand{\yRefIncrementOne}{0.0}
\newcommand{\xRefIncrementOne}{0.0}
\newcommand{\yRefIncrementTwo}{0.0}
\newcommand{\xRefIncrementTwo}{0.0}

\graphicspath{ {/home/oviazlo/Desktop/beamerPresentations/FCCee/pictures/jun19_2018/} }


\DeclareGraphicsExtensions{.eps, .pdf, .png}

\newcommand{\myBox}[2][pink] {
    \noindent\colorbox{#1}{
	\textbf{#2}
    }\par
}

% For nice block (provided by Oleh)
\tikzstyle{myBox} = [draw=red, fill=blue!1, very thick,
    rectangle, rounded corners, inner sep=5pt, inner ysep=9pt]
    
\tikzstyle{PixelBox} = [draw=PixelColor, fill=blue!1, very thick,
    rectangle, rounded corners, inner sep=5pt, inner ysep=9pt]
\tikzstyle{SCTBox} = [draw=SCTColor, fill=blue!1, very thick,
    rectangle, rounded corners, inner sep=5pt, inner ysep=9pt]
\tikzstyle{TRTBox} = [draw=TRTColor, fill=blue!1, very thick,
    rectangle, rounded corners, inner sep=5pt, inner ysep=9pt]

% poster advertisement
\newcommand{\myCenterBox}[2][pink] {
   {\centering
    \noindent\colorbox{#1}{
	\textbf{#2}
    }\par
  }
}

\newcommand{\mySmallCenterBox}[2][pink] {
   {\centering
    \noindent\colorbox{#1}{
	\textbf{{\small #2}}
    }\par
  }
}

\newcommand{\myVerySmallCenterBox}[2][pink] {
   {\centering
    \noindent\colorbox{#1}{
	\textbf{{\scriptsize #2}}
    }\par
  }
}

\newcommand{\backupbegin}{
   \newcounter{finalframe}
   \setcounter{finalframe}{\value{framenumber}}
}
\newcommand{\backupend}{
   \setcounter{framenumber}{\value{finalframe}}
}

\newcommand{\myNode}{\tikz[baseline,inner sep=1pt] \node[anchor=base]}

\tikzstyle{fancytitle} =[fill=white!15, text=black]

\definecolor{light-gray}{gray}{0.95}
% poster advertisement


\title[Calorimetry performance with CLD \hspace{14.0em}\insertframenumber/
\inserttotalframenumber]{ Calorimetry performance with CLD }


	\author[Oleksandr Viazlo]{Oleksandr Viazlo\\ 
	{\small on behalf of the CLICdp and FCC-ee collaborations}
	}
	\institute{\small CERN\\} 
	
       
	\date{22 May 2018}

% 	\logo{ \ifplacelogo \includegraphics[height=1.8cm]{./ID_week2/lund_uni-logo_s.pdf} \hspace{0.4cm} \fi}

	
%    	\frame{\titlepage}

   	

\placelogofalse


%*****************************************************************************
\begin{frame}{\large \large Kaon-pion TOF separation power with CLD}
\renewcommand{\yRefPosOne}{-0.9}
\renewcommand{\xRefPosOne}{4.2}
\renewcommand{\xRefIncrementOne}{7.5}
\begin{tikzpicture}[overlay]

 \node[inner sep=0pt] (tmp) at (\xRefPosOne+1.0,\yRefPosOne-0.1)
  {\includegraphics[width=7cm]{timing_significant.pdf}};

 \node  at (\xRefPosOne+1,\yRefPosOne+3.5) (box){%
    \begin{minipage}{1.1\textwidth}
  \begin{itemize}
   \item Separation power of the TOF for kaon-pion using timing from the outermost Outer Tracker barrel layer
   \item Kaon and pion particle gun at $\theta = $90$^{\circ}$
    \end{itemize}
    \end{minipage}
  };




 \end{tikzpicture}
\end{frame}
%*****************************************************************************

%*****************************************************************************
\begin{frame}{\large \large Muon efficiency with $b\bar{b}$ sample at 365 GeV with CLD detector}

\renewcommand{\yRefPosOne}{-1.5}
\renewcommand{\xRefPosOne}{5.3}
\renewcommand{\xRefIncrementOne}{5.5}
\begin{tikzpicture}[overlay]

%   \node  at (\xRefPosOne+0.4,\yRefPosOne+4.5) (box){%
%     \begin{minipage}{1.1\textwidth}
%   \begin{itemize}
%   \item ???
%     \end{itemize}
%     \end{minipage}
%   };

 \node[inner sep=0pt] (tmp) at (\xRefPosOne-2.7,\yRefPosOne-0.5)
    {\includegraphics[width=6cm]{muonEff_bbbar_FCCee/muonEff_bbar_vs_E.pdf}};
    
 \node[inner sep=0pt] (tmp) at (\xRefPosOne+3.3,\yRefPosOne-0.5)
    {\includegraphics[width=6cm]{muonEff_bbbar_FCCee/muonEff_bbar_vs_vertexR.pdf}};
    
 
 \node  at (\xRefPosOne+1,\yRefPosOne+3.5) (box){%
    \begin{minipage}{1.1\textwidth}
  \begin{itemize}
   \item Muon efficiency with $b\bar{b}$ sample at 365 GeV with CLD detector \\[0.3cm]
%    and 3 TeV with CLIC \\[0.3cm]
   \item Reco-truth muon matching:
   \begin{itemize}
    \item $E^{\mathrm{MC}} >$ 7.5 GeV \\[0.1cm]
    \item $\theta^{\mathrm{MC}} > $ 9$^\circ$  \\[0.1cm]
    \item $\theta(p^{\mathrm{reco}},p^{\mathrm{MC}}) < $ 1$^\circ$  \\[0.1cm]
    \item $|p_T^{\mathrm{reco}}-p_T^{\mathrm{MC}}| < $ 0.5 $p_T^{\mathrm{MC}}$  \\[0.1cm]
   \end{itemize}

   
    \end{itemize}
    \end{minipage}
  };

% % HELPER draw advanced helping grid with axises:
% \draw(-0.5,-4) to[grid with coordinates] (11.5,4);
\end{tikzpicture}
 
\end{frame}
%*****************************************************************************


\backupbegin
%*****************************************************************************
\begin{frame}
\frametitle{BACKUP} 
 
\end{frame}
%*****************************************************************************
%*****************************************************************************
\begin{frame}{\large \large Muon reconstruction algorithm}
\renewcommand{\yRefPosOne}{-0.9}
\renewcommand{\xRefPosOne}{4.2}
\renewcommand{\xRefIncrementOne}{7.5}
\begin{tikzpicture}[overlay]

 \node[inner sep=0pt] (tmp) at (\xRefPosOne-1.7,\yRefPosOne+0.3)
%   {\includegraphics[width=6cm]{singleParticleEff/CLD_muon_eff.pdf}};
  {\includegraphics[width=6cm]{../may30_2018/table_muonReco.png}};
  
 \node  at (\xRefPosOne+4,\yRefPosOne+0.2) (box){%
    \begin{minipage}{0.5\textwidth}
      \begin{itemize}
        \item First six parameters - clustering of the hits in the muon system \\[0.3cm]
        \item The next three - to define the muon system cluster candidate \\[0.3cm]
        \item The next three - to match ID track to straight line fit of muon system hits \\[0.3cm]
        \item The final six - to acossiated Calo hits to the muon PFO \\[0.3cm]
      \end{itemize}
    \end{minipage}
  };
 
 \end{tikzpicture}
\end{frame}
%*****************************************************************************
%*****************************************************************************
\begin{frame}{\large \large Event display: Pion misreconstructed as Muon}
\renewcommand{\yRefPosOne}{-0.9}
\renewcommand{\xRefPosOne}{4.2}
\renewcommand{\xRefIncrementOne}{7.5}
\begin{tikzpicture}[overlay]

 \node[inner sep=0pt] (tmp) at (\xRefPosOne+1.0,\yRefPosOne-0.1)
  {\includegraphics[width=10cm]{../may30_2018/PionMisreconstructedAsMuon.png}};

 \end{tikzpicture}
\end{frame}
%*****************************************************************************


\backupend
%********************************************************
\end{document}

