\documentclass[8pt]{beamer}

\newif\ifplacelogo % create a new conditional
\placelogotrue % set it to true

\usetheme{Warsaw}
\usecolortheme{rose}
\usepackage{multicol}
\usepackage{epstopdf}
\usepackage[italic]{hepnames}
\usepackage{tikz}
\usepackage{listings}
\usepackage{times}
\usepackage{amsmath}
\usepackage{verbatim}
\usepackage{hyperref}
\usepackage{bbding}
\lstset{breakatwhitespace,
language=C++,
columns=fullflexible,
keepspaces,
breaklines,
tabsize=3, 
showstringspaces=false,
extendedchars=true}

% TikZ includes!!!
\usepackage{tikz}
\usetikzlibrary{backgrounds}
\usetikzlibrary{calc}
\tikzstyle{every picture}+=[remember picture]
\input{/home/oviazlo/Desktop/beamerPresentations/myReports/latexHelpScripts/tikzGrid.tex}


\begin{document}

% custom colors
\definecolor{olive}{rgb}{0.3, 0.4, .1}
\definecolor{fore}{RGB}{249,242,215}
\definecolor{back}{RGB}{51,51,51}
\definecolor{title}{RGB}{255,0,90}
\definecolor{dgreen}{rgb}{0.,0.6,0.}
\definecolor{gold}{rgb}{1.,0.84,0.}
\definecolor{JungleGreen}{cmyk}{0.99,0,0.52,0}
\definecolor{BlueGreen}{cmyk}{0.85,0,0.33,0}
\definecolor{RawSienna}{cmyk}{0,0.72,1,0.45}
\definecolor{Magenta}{cmyk}{0,1,0,0}

\definecolor{PixelColor}{RGB}{207,232,139}
\definecolor{SCTColor}{RGB}{167,166,255}
\definecolor{TRTColor}{RGB}{250,224,140}
\definecolor{grayColor}{RGB}{153,153,153}

\newcommand{\yRefPosOne}{0.0}
\newcommand{\xRefPosOne}{0.0}
\newcommand{\yRefPosTwo}{0.0}
\newcommand{\xRefPosTwo}{0.0}
\newcommand{\yRefIncrementOne}{0.0}
\newcommand{\xRefIncrementOne}{0.0}
\newcommand{\yRefIncrementTwo}{0.0}
\newcommand{\xRefIncrementTwo}{0.0}

\graphicspath{ {/home/oviazlo/Desktop/beamerPresentations/FCCee/pictures/} }
\DeclareGraphicsExtensions{.eps, .pdf, .png}

\newcommand{\myBox}[2][pink] {
    \noindent\colorbox{#1}{
	\textbf{#2}
    }\par
}

% For nice block (provided by Oleh)
\tikzstyle{mybox} = [draw=red, fill=blue!1, very thick,
    rectangle, rounded corners, inner sep=5pt, inner ysep=9pt]
    
\tikzstyle{PixelBox} = [draw=PixelColor, fill=blue!1, very thick,
    rectangle, rounded corners, inner sep=5pt, inner ysep=9pt]
\tikzstyle{SCTBox} = [draw=SCTColor, fill=blue!1, very thick,
    rectangle, rounded corners, inner sep=5pt, inner ysep=9pt]
\tikzstyle{TRTBox} = [draw=TRTColor, fill=blue!1, very thick,
    rectangle, rounded corners, inner sep=5pt, inner ysep=9pt]

% poster advertisement
\newcommand{\myCenterBox}[2][pink] {
   {\centering
    \noindent\colorbox{#1}{
	\textbf{#2}
    }\par
  }
}

\newcommand{\mySmallCenterBox}[2][pink] {
   {\centering
    \noindent\colorbox{#1}{
	\textbf{{\small #2}}
    }\par
  }
}

\newcommand{\myVerySmallCenterBox}[2][pink] {
   {\centering
    \noindent\colorbox{#1}{
	\textbf{{\scriptsize #2}}
    }\par
  }
}

\newcommand{\backupbegin}{
   \newcounter{finalframe}
   \setcounter{finalframe}{\value{framenumber}}
}
\newcommand{\backupend}{
   \setcounter{framenumber}{\value{finalframe}}
}

\newcommand{\myNode}{\tikz[baseline,inner sep=1pt] \node[anchor=base]}

\definecolor{light-gray}{gray}{0.95}
% poster advertisement


\title[ Electron PID Efficiency with FCC-ee Detector \hspace{12.5em}\insertframenumber/
\inserttotalframenumber]{ Electron PID Efficiency with FCC-ee Detector }


	\author[Oleksandr Viazlo]{Oleksandr Viazlo \\ 
% 	{\small ???}
	}
	\institute{\small CERN\\} 
	
       
	\date{24 January 2018}

% 	\logo{ \ifplacelogo \includegraphics[height=1.8cm]{./ID_week2/lund_uni-logo_s.pdf} \hspace{0.4cm} \fi}

	
%    	\frame{\titlepage}

   	

\placelogofalse

%*****************************************************************************
\begin{frame}{\large \large Electron Reconstruction}
 
\renewcommand{\yRefPosOne}{0}
\renewcommand{\xRefPosOne}{5.3}
\renewcommand{\xRefIncrementOne}{5.5}
\begin{tikzpicture}[overlay]

   \node[inner sep=0pt] (tmp) at (\xRefPosOne,\yRefPosOne-1.5)
    {\includegraphics[width=8cm]{/home/oviazlo/Documents/FCCee_files/pictures/feb6/viazlo.pdf}};
 

  
\end{tikzpicture}
\end{frame}
%*****************************************************************************


%*****************************************************************************
\begin{frame}{\large \large Electron Reconstruction}
 
\renewcommand{\yRefPosOne}{0}
\renewcommand{\xRefPosOne}{5.3}
\renewcommand{\xRefIncrementOne}{5.5}
\begin{tikzpicture}[overlay]

 \node [Box] at (\xRefPosOne-0.5,\yRefPosOne) (box){%
\begin{minipage}{1\textwidth}

 \begin{itemize}
  \item Investigate track-cluster association done by Pandora to understand low efficiency for low energetic electron 
    \item Pandora calculate 3 variables which are used to decide either track is associated to cluster or not :
  \begin{itemize}
   \item Opening angle between cluster direction and track state (cos(theta) $>$ 0.0)
   \item Parallel distance between cluster and position of track state on ECAL surface (100 mm)
   \item Perpendicular distance between cluster and position of track state on ECAL surface (10 mm)
  \end{itemize}
  \item attempt to calculate these variables and check their distributions for events when electron reconstruction fails

  \end{itemize}
  
\end{minipage}
};
  
\end{tikzpicture}
\end{frame}
%*****************************************************************************



%*****************************************************************************
\begin{frame}{\large \large Events with 1 electron and 1 photon reconstructed}
 
\renewcommand{\yRefPosOne}{0}
\renewcommand{\xRefPosOne}{5.3}
\renewcommand{\xRefIncrementOne}{5.5}
\begin{tikzpicture}[overlay]

   \node[inner sep=0pt] (tmp) at (\xRefPosOne-3,\yRefPosOne+1.5)
    {\includegraphics[width=5cm]{plots_feb20/electronTrackClusterPlot1.pdf}};
 
   \node[inner sep=0pt] (tmp) at (\xRefPosOne+3,\yRefPosOne+1.5)
    {\includegraphics[width=5cm]{plots_feb20/electronTrackClusterPlot2.pdf}};
  
   \node[inner sep=0pt] (tmp) at (\xRefPosOne-3,\yRefPosOne-2.7)
    {\includegraphics[width=5cm]{plots_feb20/electronTrackClusterPlot3.pdf}};
 
   \node[inner sep=0pt] (tmp) at (\xRefPosOne+3,\yRefPosOne-2.7)
    {\includegraphics[width=5cm]{plots_feb20/electronTrackClusterPlot4.pdf}};
  
%   \node [Box] at (\xRefPosOne-0.5,\yRefPosOne-3) (box){%
% \begin{minipage}{1\textwidth}
% 
%  \begin{itemize}
%   \item Single electron (E=10 GeV) PID efficiency for all events (left) and for events w/o bremsstrahlung (right)
%   \item Pandora doesn't reconstruct electron cluster in 10-15$\%$ of events independent of theta.
%   \item No photon misreconstruction with events w/o bremsstrahlung (right plot)
%  \end{itemize}
% \end{minipage}
% };
%   
\end{tikzpicture}
\end{frame}
%*****************************************************************************

%*****************************************************************************
\begin{frame}{\large \large Events with 2 photons reconstructed}
 
\renewcommand{\yRefPosOne}{0}
\renewcommand{\xRefPosOne}{5.3}
\renewcommand{\xRefIncrementOne}{5.5}
\begin{tikzpicture}[overlay]

   \node[inner sep=0pt] (tmp) at (\xRefPosOne-3,\yRefPosOne+1.5)
    {\includegraphics[width=5cm]{plots_feb20/twoPhotons1.pdf}};
 
   \node[inner sep=0pt] (tmp) at (\xRefPosOne+3,\yRefPosOne+1.5)
    {\includegraphics[width=5cm]{plots_feb20/twoPhotons2.pdf}};
  
   \node[inner sep=0pt] (tmp) at (\xRefPosOne-3,\yRefPosOne-2.7)
    {\includegraphics[width=5cm]{plots_feb20/twoPhotons3.pdf}};
 
   \node[inner sep=0pt] (tmp) at (\xRefPosOne+3,\yRefPosOne-2.7)
    {\includegraphics[width=5cm]{plots_feb20/twoPhotons4.pdf}};
  
%   \node [Box] at (\xRefPosOne-0.5,\yRefPosOne-3) (box){%
% \begin{minipage}{1\textwidth}
% 
%  \begin{itemize}
%   \item Single electron (E=10 GeV) PID efficiency for all events (left) and for events w/o bremsstrahlung (right)
%   \item Pandora doesn't reconstruct electron cluster in 10-15$\%$ of events independent of theta.
%   \item No photon misreconstruction with events w/o bremsstrahlung (right plot)
%  \end{itemize}
% \end{minipage}
% };
%   
\end{tikzpicture}
\end{frame}
%*****************************************************************************

%*****************************************************************************
\begin{frame}{\large \large Electron efficiency}
 
\renewcommand{\yRefPosOne}{0}
\renewcommand{\xRefPosOne}{5.3}
\renewcommand{\xRefIncrementOne}{5.5}
\begin{tikzpicture}[overlay]

   \node[inner sep=0pt] (tmp) at (\xRefPosOne,\yRefPosOne)
    {\includegraphics[width=8cm]{plots_feb20/pion_eff_theta_E100.pdf}};
 
   \node [Box] at (\xRefPosOne-0.5,\yRefPosOne-3.5) (box){%
\begin{minipage}{1\textwidth}

 \begin{itemize}
  \item No effect on efficiency from changing cut on perpendicular track-cluster distance
 \end{itemize}
\end{minipage}
};
  
 
\end{tikzpicture}
\end{frame}
%*****************************************************************************

\end{document}

